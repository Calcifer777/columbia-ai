\section{Search Agents}

\paragraph{Applications}
\begin{itemize}
  \item Route-finding problem
  \item Travelling salesmane person
  \item VSLI layout: position million of components and
    connections on a chip to minimize area, shorten delays. Aim:
    put circuit components on a chip so as they don’t overlap and
    leave space to wiring which is a complex problem.
  \item Robot navigation
  \item Automatic assembly sequencing
  \item Protein folding
\end{itemize}

\subsection{Problem formulation}

\begin{itemize}
  \item Initial state: the state in which the agent starts
  \item States: All states reachable from the initial state by
    any sequence of actions (State space)
  \item Actions: possible actions available to the agent. At a
    state $s$, $Actions(s)$ returns the set of actions that can be
    executed in state $s$. (\textit{Action space})
  \item Transition model: A description of what each action does
    $Results(s, a)$
  \item Goal test: determines if a given state is a goal state
  \item Path cost: function that assigns a numeric cost to a path
    w.r.t. performance measure
\end{itemize}

\subsection{Search Space}

\begin{itemize}
  \item State space: a physical configuration
  \item Search space: an abstract configuration represented by a search tree or graph of possible solutions.
  \item Search tree: models the sequence of actions
    \begin{itemize}
      \item Root: initial state
      \item Branches: actions
      \item Nodes: results from actions. A node has: parent, children, depth, path cost, associated state in the state space.
    \end{itemize}
  \item Expand: A function that given a node, creates all children nodes
\end{itemize}

The search space is divided into three regions:
\begin{itemize}
  \item Explored (a.k.a. Closed List, Visited Set)
  \item Frontier (a.k.a. Open List, the Fringe)
  \item Unexplored.
\end{itemize}

\subsection{Search Strategies}

\paragraph{Strategy evaluation} Strategies are evaluated along
the following dimensions:
\begin{itemize}
  \item Completeness: Does it always find a solution if one
    exists?
  \item Time complexity: Number of nodes generated/expanded
  \item Space complexity: Maximum number of nodes in memory
  \item Optimality: Does it always find a least-cost solution?
\end{itemize}

Time and space complexity: are measured in terms of:
\begin{itemize}
  \item $b$: maximum branching factor of the search tree (actions
    per state).
  \item $d$: depth of the solution
  \item $m$, or $D$: maximum depth of the state space (may be
    $\infty$).
\end{itemize}
